\chapter{Additional examples}
\section{Code listings}
\lstset{
    basicstyle = \ttfamily,
    keywordstyle = \color{blue}\textbf,
    commentstyle = \color{gray},
    stringstyle = \color{green!70!black},
    stringstyle = \color{red},
    numbers=left,
    numbersep=5pt,
    numberstyle = \scriptsize\sffamily\color{gray},
    showstringspaces = false,
}
Here is a portion of the bash build script, using the \verb|listings| package.

\begin{lstlisting}[language=bash,caption={Portion of build.sh script.},label={lst:build}]
if [ $clean -eq 1 -a -e $outdir ]
then
    if [ $DEBUG -eq 1 ]
    then
        echo "Removing output directory ${outdir}."
    fi
    rm -rf $outdir
fi

if [ $pdf -eq 1 ]
then
    if [ $DEBUG -eq 1 ]
    then
        echo "Building PDF."
    fi
    build_pdf "${outdir}/pdf"
fi
\end{lstlisting}

Here is a code listing that is loaded from an external file.

\lstinputlisting[language=logo,caption={Drawing a hypotrochoid in Logo.},label={lst:star}]{content/main/more/star.logo}

\section{\TeX\ equation listing and output}

Here is \TeX\ code for the equation that appears in \autocite{lam2017}.

\lstinputlisting[language={[LaTeX]TeX},caption={\TeX\ code for a mathematics expression.},label={lst:example}]{content/main/more/eq.tex}

\begin{equation}
    \lambda \mathrel{\mathop:}= \lim_{x_1 \to \infty}
    \int_{x_0}^{x_1}
    \frac{f(\frac{t}{2})}{\sqrt[n]{t^2 + \sin^2(t)}}
    \, dt \stackrel{!}{\leq} 1
\end{equation}\label{eqn:example}


\section{\TeX\ table listing and output}

How about a table? Here are the ten most popular Linux distributions according to \url{distrowatch.com}
for the first half of 2022 \autocite{dis2022}.

\lstinputlisting[language={[LaTeX]TeX},caption={\TeX\ code for a colorful table.},label={lst:table}]{content/main/more/distrowatch.tex}

\begin{table}[htbp!]
    \centering
    \begin{tabular}{rlr}
        \rowcolor{blue}
        \textcolor{white}{\textbf{Rank}} &
        \textcolor{white}{\textbf{Distribution}} &
        \textcolor{white}{\textbf{Hits per Day}} \\
        \rowcolors{2}{gray!15}{white}
        1 & MX Linux & 2920 \\
        2 & EndeavourOS & 2684 \\
        3 & Mint & 2227 \\
        4 & Manjaro & 1721 \\
        5 & Pop!\_OS & 1373 \\
        6 & Ubuntu & 1327 \\
        7 & Fedora & 1165 \\
        8 & Debian & 1075 \\
        9 & Garuda & 1030 \\
        10 & Zorin & 835 \\
    \end{tabular}
    \caption{Top ten Linux distributions -- 1H 2022.}
    \label{tab:distros}
\end{table}

