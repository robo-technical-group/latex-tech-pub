\newglossaryentry{tex}{
    name = {\TeX},
    sort = {TEX},
    description = {Sophisticated digital typesetting system,
    famous for high typographic quality of
    mathematical formulae.
    }
}

\newglossaryentry{latex}{
    name = {\LaTeX},
    sort = {LATEX},
    description = {Document markup language based on
    \gls{tex}, widely used in academia.
    }
}

\newglossaryentry{tikz}{
    name = {Ti\emph{k}Z},
    sort = {TikZ},
    description = {Extremely capable graphics language for drawing with \gls{tex}.
    }
}

\newglossaryentry{foreword}{
    name = {foreword},
    description = {The foreword is an introduction to a book that typically
        is written by someone other than the book's author(s).}
}

\newglossaryentry{introduction}{
    name = {introduction},
    description = {In the introduction, the author gives a brief overview of the book,
        perhaps with a summary of each chapter.}
}

\newglossaryentry{preamble}{
    name = {preamble},
    description = {In a \gls{latex} document, the preamble appears at the beginning
        of the document and consists of the code that appear before
        \verb|\begin{document}|.}
}

\newglossaryentry{preface}{
    name = {preface},
    description = {The preface of a book describes why the author has written it.}
}

\newglossaryentry{colophon}{
    name = {colophon},
    description = {The colophon is a brief statement about the publication of a book,
    including information like the name of the publisher and fonts used.
    From the Greek \emph{kolophōn},
    meaning \enquote{summit} or \enquote{finishing touch} \autocite{oup2021}.}
}

\newglossaryentry{saddlepoint}{
    name = {saddle point},
    description = {A point on a three-dimensional surface where
    tangental slopes (i.e., derivatives) in the orthogonal directions are zero, but the point
    itself is not a local extremum.}
}

\newglossaryentry{minimax}{
    name = {minimax point},
    description = {See \emph{\gls{saddlepoint}}.}
}

